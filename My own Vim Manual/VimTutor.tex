\documentclass[11p]{book}
\usepackage{amsmath,mathtools}
\usepackage[utf8]{inputenc}
\usepackage[spanish]{babel}
\usepackage{graphicx}
\usepackage{geometry}
    \geometry{a4paper,total={210mm,297mm},left=30mm,right=20mm,top=30mm,bottom=30mm,}
% \usepackage{upquote}% getting the right grave ` (and not ‘)!

\begin{document}
These are notes of $Vim$ taken by myself. Not all commands are listed in this list.
\chapter*{Vim Commands List}
\newpage

\section*{Rare or unusual commands}
\begin{tabular}{p{4cm} l}
CTRL-] & to jump to a subject under the cursor\\
CTRL-O & to jump back (repeat to go further back)\\
\hline
\end{tabular}
\newpage

\section*{Most used commands}
The cursor moving is achieved by pressing the hjkl keys:
\begin{itemize}
\item h $-->$ move to the left
\item j $-->$ move to the bottom
\item k $-->$ move to the top
\item l $-->$ move to the right
\end{itemize}

That is

\hspace*{8cm}k \\
\hspace*{8.5cm}$\wedge$\\
\hspace*{8.5cm}$\>|$\\
\hspace*{7.32cm}h $<$ - - $\>|$ - - $>$ l \\
\hspace*{8.5cm}$\vee$ \\
\hspace*{8.6cm}j \\

By pressing $<ESC>$ Vim goes to Normal mode. \\

By writing $$operator \quad [number] \quad motion$$ the operation $operator$ is repeated $[number]$ times the $motion$. 

"$!$" \hspace*{0.1cm} is the override command modifier.

"$:help$" gives you generic help. \\

\begin{tabular}{p{2cm} p{13cm}}
i & to start the Insert mode \\
a & to start the Insert mode at the consecutive cursor's position\\
v & to start the Visual mode \\
x & to delete the character where the cursor is \\
J & joins the next line with the actual one, which the cursor is in and it isn't need to be it at the end \\
CTRL-R & undoes the undo \\
CTRL-O & undoes de undoing of undo \\
o & opens a new empty line below the cursor and puts $Vim$ in Insert mode \\
O(uppercase) & opens a new empty line above the cursor in Insert Mode \\
ZZ & This writes the file and exits. \\
q! & Quit and throw things away command, whitout saving changes. \\
e! & reloads the original version of the file \\
\hline
w & moves to the beginning of next word \\
b & moves to the previous beginning word \\
e & moves to the next end word \\
ge & moves to the previous end word \\
\end{tabular} \\

%\vspace*{1cm}
$<---<--<-<------<--$ \hspace*{1cm} $--->-->->----------->$ \\
\hspace*{1.5cm} b \hspace{0.78cm}b \hspace*{0.5cm} b \hspace*{1cm} 2b \hspace*{1.2cm} b \hspace*{1.5cm} w \hspace*{0.8cm} w \hspace*{0.4cm} w \hspace*{1.5cm} 3w \\

\hspace*{4.2cm}$<--<---$ \hspace*{1cm } $---->--->$ \\
\hspace*{5.2cm} ge \hspace*{0.6cm} ge  \hspace*{2cm} e \hspace*{1cm} e\\


\begin{tabular}{p{2cm} p{13cm}}
W(uppercase) & same than $w$, but jumps between spaces \\
E(uppercase) & same than $e$, but jumps between spaces \\
B(uppercase) & same than $b$, but jumps between spaces \\
gE(uppercase) & same than $ge$, but jumps between spaces \\
\hline
0(cero) & The cursor is moved to the beginning of the line \\
"$\wedge$" & Moves the cursor to the first non-black character of the line. \\
\$ & Locates the cursor at the end of the line. \\ 
$<number>$\$ & Same as \$ but at the $<number>-th$ below the actual one.\\

\end{tabular}

\newpage
\section*{Set commands}
An obtion with a "no" at its begining it's set to off. \\

\begin{tabular}{p{4cm} l}
:set showmode & to be able to see the actual mode, the one you're in \\

\end{tabular}
\end{document}